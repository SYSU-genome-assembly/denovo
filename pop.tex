\documentclass[a4paper]{article}

\usepackage{xeCJK}
\usepackage[backend=biber]{biblatex}

\addbibresource{elex.bib}

\setCJKmainfont{simsun.ttf}
\setCJKsansfont{simhei.ttf}
\setCJKmonofont{simfang.ttf}

\setlength{\parindent}{2em}
\setlength{\parskip}{.5em}

\title{基于Leslie模型的放开二胎政策后的\\人口学预测及相关分析}
\author{张炜星\, 陶冠兰\, 胡乃超}
\date{May, 2014}

\begin{document}
\maketitle
\begin{abstract}
自计划生育政策以来,我国妇女生育水平逐渐降至更替水平以下,并长期维持
在较低水平,引发了对生育政策调整的热烈讨论。文章利用多次全国1\%人口
抽样调查人口数据及普查数据,推算2012年我国独生子女的规模,并从独生子
女的角度估算2012年立即全面放开二胎政策的目标人群,再结合妇女生育二胎
的意愿所处水平,估算未来人口增长的趋势。并从所建的模型提供一些社会学分析。\par
\emph{关键词: }放开二胎政策; 独生子女; 总和生育率; 人口预测
\end{abstract}
\tableofcontents

\section{问题重述}
\subsection{原题}
人口的数量和结构是影响经济社会发展的重要因素。从20世纪70年代后期以来
,我国鼓励晚婚晚育,提倡一对夫妻生育一个孩子。该政策实施30多年来,有
效地控制了我国人口的过快增长,对经济发展和人民生活的改善做出了积极的
贡献。但另一方面,其负面影响也开始显现。如小学招生人数(1995年以来)
、高校报名人数(2009年以来)逐年下降,劳动人口绝对数量开始步入下降通
道,人口抚养比的相变时刻即将到来,这些对经济社会健康、可持续发展将产
生一系列影响,引起了中央和社会各界的重视。党的十八届三中全会提出了开
放单独二孩,今年以来许多省、市、自治区相继出台了具体的政策。政策出台
前后各方面人士对开放“单独二孩”的效应有过大量的研究和评论。\par
人口问题有着悠久的研究历史,也有不少经典的理论和模型。这些理论和模型
都依赖生育模式、生育率、死亡率和性别比等多个因素。这些因素与政策及人
的观念、社会文化习俗有着紧密的关系,后者又受社会经济发展水平的影响。
研究中用到的数据的置信水平也与调查统计有关。\par
请收集一些典型的研究评论报告,根据每十年一次的全国人口普查数据,建立
模型,对报告的假设和某些结论发表自己的独立见解,并针对深圳市或其他某
个区域,讨论计划生育新政策(可综合考虑城镇化、延迟退休年龄、养老金统
筹等政策因素,但只须选择某一方面作重点讨论)对未来人口数量、结构及其
对教育、劳动力供给与就业、养老等方面的影响。
\subsection{我们要写的问题重述(或者叫【引言】也行)}
%下面的依然是直接抄cite的那篇,反正综合几篇写一个出来就好了
2013年11月15日,中共中央十八届三中全会审议通过中共中央关于全面深化改革
若干重大问题的决定,其中明确的提出了启动实施一方是独生子女的夫妇可生育
两个孩子的政策,逐步调整完善生育政
策。由此,在我国严格实施了三十余年的计划生育政策迎来了一轮新的完善与调整。
在社会各界欢迎这一被俗称为“单独二胎”\footnote{``单独''即夫妻中一方是
  独生子女,系与夫妻双方均为独生子女的``双独''相对而言,相当于专有名
  词,下同。}的政策的同时,学术界以及部分社会大众提出新的疑问,计划生
育新政策在综合考虑城镇化、延迟退休年龄、养老金统筹等政策
因素时会对未来人口数量、结构及其对教育、劳动力供给与就业、养老等方面产
生怎样的影响。\par
%似乎主要来自于 zhenwu14 那篇的提到的文,关于政策完善讨论的文
而要尽量准确回答这些问题首先必须把握符合单独二孩的目标人群年龄分布和结
构,并对调整政策后人口的结构及数量变化量化分析。
学界关于此方面的讨论并非始于2013年,而是近十年来一直持续展开着比较激烈
的讨论。乔晓春等\cite{xiaochun06}估计了维持现行政策以及放开生育政策两
种情形下的人口学后果,结合国外的经验教训,从低生育率的长期经济效应及社
会后果分析了如何合理的调整生育政策。陈友华\cite{youhua07}则指出虽然政
策调整导致的生育率反弹是不可避免的,但是政策变动带来的出生堆积可以通过
相应对策加以缓和。因此对于总和生育率已降至更替水平以下的我国应尽快对生
育政策进行调整。另一方面郭震威\cite{zhenwei09}通过比较现行政策不变、
``单独''政策和``二孩''政策三种不同方案人口和经济社会后果,认为保持现行
生育政策不变或为较好选择。更为近期的文献则更多针对新政策的后果进行了分
析,如王广州等\cite{guangzhou12}通过人口系统随机微观仿
真实验分析,假定2015年全国城乡统一放开二孩政策,出生人口堆积将增加600
万左右,出生人口规模在2100万左右,总结实施放开``单独二孩''政策面临的风
险不大。曾毅\cite{zengyi12}则表示应尽快允许普遍生育二孩,而只开放``单
独二孩''并不明智,2013年为二孩政策方案启动的最佳时
间。显然不同学者对政策变动带来的人口后果意见并不统一,争论的焦点之一是
生育政策的变动究竟会给出生人口带来多大的影响? 会造成多大程度的人口出生
堆积,给未来教育医疗就业养老等造成多大的冲击? 生育政策变动后,我国人口
的未来走势如何?\par
\section{符号说明及基本假设}
\subsection{符号说明}
表\ref{tab:marks}中所列出的符号为本文模型推导主要使用的符号,即可以直接从所用数据中读取或可以从数据及假定中估算出来的数量。而其他一些相对间接变量或这中间变量则会在用到时做出相应的补充说明。另外由于采用Leslie矩阵模型,$t$为离散变量,且为简化模型取2008年时$t=0$,即$t=0,1,2,\ldots$。
\begin{table}[h!]
\centering
\begin{tabular}{c c}
 \hline
 变量含义 & 对应符号 \\ [0.5ex]
 \hline
 第t年初年龄为i的总人数 & $P_i(t)$ \\
 第t年末年龄为i的人的死亡率 & $\mu_i(t)$ \\
 第t年末平均每个育龄女性的生育数 & $\beta(t)$ \\
 生育加权因子(即生育模式)& $h(t)$ \\
 第t年末生育性别比 & $\sigma(t)$ \\
 第t年末i岁人口中女性比例 & $F_i(t)$ \\
 独生子女比例 & $a(t)$ \\
 第t年意愿生育概率向量 & $\textbf{p(t)}$ \\ [.5ex]
 \hline
\end{tabular}
\caption{符号说明}
\label{tab:marks}
\end{table}
\subsection{基本假设}%从07那篇直接抄的
\begin{enumerate}
\item 每一年的人口总数,人口结构及分布和其他有关各量仅在年末发生变化,变化顺序是:一部分人先死亡,然后一部分人生小孩,最后一部分人迁移。
%上面的顺序有待讨论
\item 本文中所提到的婴儿出生率指的是婴儿出生且在一岁前存活的概率。
\item 生育妇女一年只生一胎。百岁以上的人口变化对总人口变化影响不大,因此不予以考虑。
\item 人口的迁移路径仅考虑从村到镇,从村到城。
%上面这条有待讨论
\item 国际迁入迁出对于人口的影响较小。
%下面这条自己加的,我觉得这样可以保持p值函数不变
\item 执行新政策前后,所有妇女生育前后心理状态相同。
\end{enumerate}
\section{问题分析}
% 对其他研究报告的评论塞在这里好了

%关于总和生育率
在以往研究中,人口学家对这一问题展开了讨论 。乔晓春、 任强( 2006) 在他们的研究中直接假设
放开生育政策后妇女总和生育率将线性上升到更替水平左右,这种方法仅仅是学者根据自身的人口
学知识以及一些特定的假定做出的粗略估算,没有考虑历史累积的二胎生育能量在生育政策变动后
的堆积释放等关键问题 。郭志刚( 2004) 认识到传统的人口预测方法在进行生育政策调整模拟时存在
很多不足与局限,传统的预测方法不能控制育龄妇女本身的孩次结构影响 ,因此他提出可以将年龄递
进生育模型应用到生育政策调整研究中 。王广州( 2011) 按照胎次递进比模型,估算了北京市生育政
策调整后分城乡人口总和生育率的变动 ,并测算了对年度出生人口规模的政策影响 。2013 年,他和胡
耀岭等人进一步在全国性分析中采用胎次递进比方法,测算全面放开二胎后出生人口及总人口的变
动趋势。胎次递进比的方法一定程度上弥补了传统人口预测在生育政策调整模拟中的不足,但是这
种方法计算相对复杂,且需要详细的妇女孩次结构数据,测算结果的准确性在很大程度上受到妇女生
育数据质量的影响 。
\cite{zhenwu14}\par
\section{模型建立与求解}
\subsection{数据来源}
\subsection{政策前Leslie模型}
\subsection{独生子女规模及向量\textbf{p(t)}的估算}
\subsection{新政策下的修正模型}
\section{结论与讨论}
\medskip
\printbibliography %print出来就是references
\end{document}
