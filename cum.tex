\documentclass[12pt,a4paper]{article}

\usepackage{xeCJK}
\usepackage{amsmath}
\usepackage{balance}
\usepackage{graphicx}
\usepackage{wrapfig}
\usepackage{subcaption}
\graphicspath{ {images/} }
%\usepackage[backend=biber,
%style=phys,]{biblatex}

\usepackage[top=1.25in,bottom=1.25in,left=1in,right=1in]{geometry}

%\addbibresource{elex.bib}


\setCJKmainfont{simsun.ttf}
\setCJKsansfont{simhei.ttf}
\setCJKmonofont{simfang.ttf}

\setlength{\parindent}{2em}
\setlength{\parskip}{.5em}

\title{创意平板折叠桌参数模型}

\author{}
\date{}

\begin{document}

\maketitle
\renewcommand{\abstractname}{摘要}
\begin{abstract}
前两页官方承诺xxx,据说占满第三页的摘要。\par
摘要。摘要。摘要。摘要。摘要。摘要。摘要。摘要。摘要。摘要。摘要。摘要。摘要。摘要。摘要。摘要。摘要。摘要。摘要。摘要。摘要。摘要。摘要。摘要。摘要。摘要。摘要。摘要。摘要。摘要。摘要。摘要。摘要。摘要。摘要。摘要。摘要。摘要。摘要。摘要。摘要。摘要。摘要。摘要。摘要。摘要。摘要。摘要。摘要。摘要。摘要。摘要。摘要。摘要。摘要。摘要。摘要。摘要。摘要。摘要。摘要。摘要。摘要。摘要。摘要。摘要。摘要。摘要。摘要。摘要。摘要。摘要。摘要。摘要。摘要。摘要。摘要。摘要。摘要。摘要。摘要。摘要。摘要。摘要。摘要。摘要。摘要。摘要。摘要。摘要。摘要。摘要。摘要。摘要。摘要。摘要。摘要。摘要。摘要。摘要。摘要。摘要。摘要。摘要。摘要。摘要。摘要。摘要。摘要。摘要。摘要。摘要。摘要。摘要。摘要。摘要。摘要。摘要。摘要。摘要。摘要。摘要。摘要。摘要。摘要。摘要。摘要。摘要。摘要。摘要。摘要。摘要。摘要。摘要。摘要。摘要。摘要。摘要。摘要。摘要。摘要。摘要。摘要。摘要。摘要。摘要。摘要。摘要。摘要。摘要。摘要。摘要。摘要。摘要。摘要。摘要。摘要。摘要。摘要。摘要。摘要。摘要。摘要。摘要。摘要。摘要。摘要。摘要。摘要。摘要。摘要。摘要。摘要。摘要。摘要。摘要。摘要。摘要。摘要。摘要。摘要。摘要。摘要。摘要。摘要。摘要。摘要。摘要。摘要。摘要。摘要。摘要。摘要。摘要。摘要。摘要。摘要。摘要。摘要。摘要。摘要。摘要。摘要。摘要。摘要。摘要。摘要。摘要。摘要。摘要。摘要。摘要。摘要。摘要。摘要。摘要。摘要。摘要。摘要。摘要。摘要。摘要。摘要。摘要。摘要。摘要。摘要。摘要。摘要。摘要。摘要。摘要。摘要。摘要。摘要。摘要。摘要。摘要。摘要。摘要。摘要。摘要。摘要。摘要。摘要。摘要。摘要。摘要。摘要。摘要。摘要。摘要。摘要。摘要。摘要。摘要。摘要。摘要。摘要。摘要。摘要。摘要。摘要。摘要。摘要。摘要。摘要。摘要。摘要。摘要。摘要。摘要。摘要。摘要。摘要。摘要。摘要。摘要。摘要。摘要。摘要。摘要。摘要。摘要。摘要。摘要。摘要。摘要。摘要。摘要。摘要。摘要。摘要。摘要。摘要。摘要。摘要。摘要。摘要。摘要。摘要。摘要。摘要。摘要。摘要。摘要。摘要。摘要。摘要。摘要。摘要。摘要。摘要。摘要。摘要。摘要。摘要。摘要。摘要。摘要。摘要。摘要。摘要。摘要。摘要。摘要。摘要。摘要。摘要。摘要。摘要。摘要。摘要。摘要。摘要。摘要。摘要。摘要。摘要。摘要。摘要。摘要。摘要。摘要。摘要。摘要。摘要。摘要。摘要。摘要。摘要。摘要。摘要。摘要。摘要。摘要。摘要。摘要。摘要。摘要。摘要。摘要。摘要。摘要。摘要。摘要。摘要。摘要。摘要。摘要。摘要。摘要。\par
\emph{关键词:}创意平板折叠桌;客户就是上帝;结果还是选了B题
\end{abstract}


\renewcommand{\contentsname}{目录}
\tableofcontents

\section{问题重述}
某公司生产一种可折叠的桌子,桌面呈圆形,桌腿随着铰链的活动可以平摊成一张平板。桌腿由若干根木条组成,分成两组,每组各用一根钢筋将木条连接,钢筋两端分别固定在桌腿各组最外侧的两根木条上,并且沿木条有空槽以保证滑动的自由度。桌子外形由直纹曲面构成,造型美观。附件视频展示了折叠桌的动态变化过程。\par
试建立数学模型讨论下列问题:\par
1. 给定长方形平板尺寸为$120 cm \times 50 cm \times 3 cm$,每根木条宽2.5 cm,连接桌腿木条的钢筋固定在桌腿最外侧木条的中心位置,折叠后桌子的高度为53 cm。试建立模型描述此折叠桌的动态变化过程,在此基础上给出此折叠桌的设计加工参数和桌脚边缘线的数学描述。\par
2. 折叠桌的设计应做到产品稳固性好、加工方便、用材最少。对于任意给定的折叠桌高度和圆形桌面直径的设计要求,讨论长方形平板材料和折叠桌的最优设计加工参数,例如,平板尺寸、钢筋位置、开槽长度等。对于桌高70 cm,桌面直径80 cm的情形,确定最优设计加工参数。\par
3. 公司计划开发一种折叠桌设计软件,根据客户任意设定的折叠桌高度、桌面边缘线的形状大小和桌脚边缘线的大致形状,给出所需平板材料的形状尺寸和切实可行的最优设计加工参数,使得生产的折叠桌尽可能接近客户所期望的形状。你们团队的任务是帮助给出这一软件设计的数学模型,并根据所建立的模型给出几个你们自己设计的创意平板折叠桌。要求给出相应的设计加工参数,画出至少8张动态变化过程的示意图。\par
\cite{qiyuan03}
\cite{guangzhou12}
\cite{guangzhou09}

%话说“论文不能有页眉或任何可能显示答题人身份和所在学校等的信息。”是说上面不%能写作者名字的么……
%大概第一二页写了就不用在正文里写了吧

\section{通用符号说明及假设}
本文符号说明如表\ref{t:notas}所示。主要假设如下:
\begin{enumerate}
\item 该公司建造桌子所用板材无厚度之分,统一使用厚度为$3cm$的木板;
\item 该公司对木板的处理方法一致,所得木条宽度均为$2.5cm$;
\item 木桌上木条边缘的边中点满足所需几何图形方程时认为木桌与该图形相符;
\item 
\end{enumerate}

\renewcommand{\tablename}{表}
\begin{table}[h!]
\centering
\begin{tabular}{c c}
 \hline
 变量含义 & 对应符号 \\ [0.5ex] 
 \hline
 桌子参数尺寸 & $2a\,cm \times 2b\,cm \times c\,cm$ \\
 木条总数量 & $n$ \\
 桌腿实际长度 & $L$ \\
 桌子高度 & $h$ \\
 第t年内生育性别比 & $\sigma(t)$ \\
 第t年内i岁人口中女性比例 & $F_i(t)$ \\
 独生子女比例 & $a(t)$ \\
 第t年内意愿生育概率向量 & $\textbf{p(t)}$ \\ [.5ex]
 \hline
\end{tabular}
\caption{符号说明}
\label{t:notas}
\end{table}

\renewcommand{\figurename}{图}
\begin{figure}
\centering
\includegraphics{topp}
\caption{据说该换图……}
\label{fig:my_label}
\end{figure}

\section{问题分析}
%不如这里塞连续型的方程?
%夹菌说好的用小标题概括结论?
\section{模型建立与求解}
\subsection{问题一:给定尺寸的圆面折叠桌动态模型}
\subsection{问题二:圆面折叠桌的参数优化模型}
\subsection{问题三:任意版面折叠桌的参数优化模型}
\section{模型敏感性分析}
\section{结论及讨论}

\medskip
\renewcommand\refname{参考文献}
%\printbibliography %print出来就是references
\bibliographystyle{unsrt}
\bibliography{elex}

\end{document}

