\documentclass[a4paper]{article}

\usepackage{xeCJK}
\usepackage{authblk}

\setCJKmainfont{simsun.ttf}
\setCJKsansfont{simhei.ttf}
\setCJKmonofont{simfang.ttf}

\title{基于 reads 引导的基因组序列拼接 }
\author{张炜星 陶冠兰 胡乃超}
\date{May, 2014}

\begin{document}
\maketitle
\begin{abstract}
新一代测序技术的快速发展,为生命科学重大问题的研究提供巨大帮助的同时,
其数据海量、读取片段 reads 长度短、精确度低等特点也为全基因组序列拼接
提出了相当严峻的挑战,而依靠传统的序列拼接算法已很难有效应对。针对新一
代测序的数据特点,研发效率更高的基因组序列拼接算法,已显得极为迫切。
本文首先总结了当前研究中不同算法的优势及不足,并提出了序列拼接算法的改
进方向。接着,基于改进方向提出基因组序列拼接算法,并选取若干组数据对该
算法软件进行测试。\\*
\vspace{0.4cm}

\textbf{关键词:} 新一代测序技术;基因组序列拼接;欧拉路径;De Bruijn 图; k-mer

\end{abstract}

\section{引言}

\section{DNA序列拼接的研究进展}

\section{改进基于xx的DNA序列拼接算法}

\section{算法验证及性能分析}

\section{结论}

\section*{参考文献}
\section*{附录}
\end{document}
